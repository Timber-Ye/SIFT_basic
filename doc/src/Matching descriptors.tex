\part{Using SIFT for Image Matching}\label{part:Matching descriptors}

基于前面提取\sift 特征点、描述\sift 特征点的过程,本部分进一步介绍如何进行多视角图像间特征点的匹配,并展示一个最终的实验结果。

\section{描述子匹配}

寻找一幅图像中某个特征点在另一幅图像中的对应,即寻找与之描述子最为接近的特征点。衡量两个描述子(即特征向量)之间距离的方式有很多,例如欧氏距离(2-范数),汉明距离等。这里,我直接使用欧氏距离,并且以暴力匹配的方式寻找描述子之间的匹配关系。

同时,不一定一幅图像中的每一个描述子都能在另一幅图像中找到对应点,因此可以对描述子之间的距离设置一个上限;并且可以每次选取距离最近的两个描述子,如果最接近的描述子距离上要明显小于次接近的描述子,则可认为最接近的描述子是可靠的。

\section{实验效果:图像特征匹配}
