\part{Summary}\label{part:Summary}

\sift 是一种在计算机视觉领域中传统而又常用的图像处理算法。SIFT算法的主要意义在于其能够有效地提取图像中的特征,并通过特征匹配来实现图像的识别、匹配、定位等应用。最后再总结一下其最为基本的五大步骤:
\begin{enumerate}
    \item \textbf{构建尺度空间} \quad \lpl 所具有的空间选择特性为特征的尺度分析提供了方法。\sift 当中用高斯差分近似\lpl 以高效构建尺度空间(\equref{equ: DoG and LoG})。其具体过程首先构建了高斯金字塔,然后在每个尺度上,通过相邻两个高斯模糊后的图像相减,得到差分图像。将每个尺度的差分金字塔中相同位置的差分图像组合成一个3D张量,构成尺度空间。每个张量代表了在一个特定尺度下图像的局部特征。在构建尺度空间的过程中,\octave 的层数$o$,每层\octave 包含的尺度个数$S$,以及高斯差分函数的$\sigma$参数是三个重要的经验参数,它们决定了所能检测到的尺度范围,以及每个尺度之间的差异程度。为了保证图像的尺度不变性,它们的值需要根据特征尺度进行自适应调整,以便提取到不同尺度下的关键点。
    \item \textbf{定位关键点} \quad 定位关键点的过程就是进行非极大值抑制的过程,即既要保证特征在某个高斯差分上的响应超过阈值,还要保证其在 \DoG 尺度空间比它的领域都要大。首先,对每个尺度的差分金字塔中的每个像素,如果其响应值超过了阈值,则继续检查其周围3x3x3邻域内的26个像素,确定其是否为该邻域的极值点。对于每个尺度的差分金字塔,将极值点的坐标和所处的尺度记录下来,形成关键点集合。
    \item \textbf{主方向分配} \quad 分析以关键点为中心的局部图像块的梯度主方向。图像的梯度分析对光照变化更为鲁棒,而梯度主方向的确定可以消除特征的旋转不确定性。按\equref{equ: compute gradient}方式计算以每个\sift 特征关键点为中心提取出$16\times 16$的局部图像块的梯度。接下来通过构建梯度方向直方图的方式确定主方向。在找到这个主方向之后,通过对关键点周围的图像区域进行旋转的方式, 将该主方向变换到统一的标准方向。
    \item \textbf{特征描述} \quad 每个关键点生成一个具有区分度和鲁棒性的特征描述符,用于表示关键点周围的图像信息。首先将关键点所在的尺度空间的图像分为$4\times 4 =16$个大小相等的子区域,为每个子区域构建长度为$8$的梯度方向直方图。将这$16$个梯度方向直方图拼接在一起后进行归一化,得到最终的$128$维特征向量。
    \item \textbf{描述子匹配} \quad 通过比较两张图像的特征描述符,找出相似的特征点,实现图像间的匹配和识别。可直接使用欧氏距离,并且以暴力匹配的方式寻找描述子之间的匹配关系。
\end{enumerate}

实验结果表明,所实现的\sift 算法的特征匹配正确率$77.8\%$,能够较为鲁棒地解决尺度、旋转等变化条件下的特征匹配任务。另外,实验中出现的两对错误匹配也表明,仅仅依靠局部特征描述无法区分重复的纹理特征。

可以看出,\sift 特征提取与匹配算法有着清晰而又复杂的处理流程,并且依靠许多经验参数,但这还只是最为基础的一种实现方式。将来在此基础还可以做出如下改进:
\begin{itemize}
    \item \textbf{关键点的精确定位}\quad 在离散采样中搜索到的极值点不一定是真实空间的极值点,因此对尺度空间DoG函数进行泰勒展开,在近似连续域上计算其极值点,从而实现关键点的精确定位。
    \item \textbf{增加关键点对视角的鲁棒性}\quad 对视角变化具有一定的鲁棒性,但是在大视角变化下其性能会受到影响。由于透视变换的作用,在数据库图像上原本接近圆形的特征会在查询图像上畸变为椭圆,从而导致无法匹配。\cite{mikolajczyk2004scale}中提出了基于二阶协方差矩阵的仿射不变特征点检测算法。
    \item \textbf{去除边缘上的点}\quad 由于DoG对图像中的边缘有比较强的响应值,而一旦特征点落在图像的边缘上,这些点就是不稳定的点。它们很难定位,具有定位歧义性,且容易受到噪声的干扰而变得不稳定。根据主曲率比值来判断关键点是否为稳定的特征点,如果不稳定则删除。
\end{itemize}
