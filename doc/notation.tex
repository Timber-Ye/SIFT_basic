\newcommand\figref[1]{\textbf{图}~\ref{#1}}
\newcommand\thmref[1]{\textbf{定理}~\ref{#1}}
\newcommand\equref[1]{\textbf{式}~(\ref{#1})~}
\newcommand\secref[1]{\textbf{第}~\ref{#1}~\textbf{节}}
\newcommand\parref[1]{\emph{Part~\ref{#1}}}
\newcommand{\zhparen}[1]{(\raisebox{0.1ex}{#1})}
\newcommand{\enparen}[1]{\CASE{(}#1\CASE{)}}

\newcommand{\drawgrid}{ % 绘制辅助网络,坐标范围:(0, 0)--(1,1)
    \draw[very thin, draw=gray, step=0.02] (0,0) grid (1,1);
    \draw[thin, draw=red, xstep=0.1, ystep=0.1] (0,0) grid (1,1);

    \foreach \x in {0,1,...,9} { \node [anchor=north] at (\x/10,0) {\tiny 0.\x}; }
    \node [anchor=north] at (1,0) {\tiny 1};

    \foreach \y in {0,1,...,9} { \node [anchor=east] at (0,\y/10) {\tiny 0.\y}; }
    \node [anchor=east] at (0,1) {\tiny 1};
}

\renewcommand{\figurename}{图}

\newmintinline{python}{breakbytoken=true}
\newmintedfile{python}{breakanywhere,breaklines, linenos, mathescape, frame=none}
\newmintedfile{text}{breakanywhere,breaklines, linenos, mathescape, frame=none}

\newminted{python}{linenos, breakanywhere,breaklines, bgcolor=bg, baselinestretch=1.2, firstnumber=last, python3}
\newminted{zsh}{breakanywhere,breaklines, bgcolor=bg, baselinestretch=1.2}
\newminted{yaml}{linenos, breakanywhere,breaklines, bgcolor=bg, baselinestretch=1.2, showspaces=true, space=·}

\colorlet{color_mark}{Maroon}
\colorlet{color_markbox}{Maroon}

\colorlet{keyword}{松花绿}
\colorlet{comment}{漆黑!50}
\colorlet{texcs}{酡红}
\colorlet{emph1}{靛蓝}
\colorlet{emph2}{琥珀}
\colorlet{inline}{玄色}

\definecolor{building}{RGB}{255, 224, 138}
\definecolor{forest}{RGB}{161, 228, 175}
\definecolor{water}{RGB}{101, 179, 255}
\definecolor{misc}{RGB}{225, 104, 219}
\definecolor{unlabeled}{RGB}{102, 102, 102}
\definecolor{bg}{rgb}{0.95,0.95,0.95}

% \newacronym{⟨label⟩}{⟨abbrv⟩}{⟨full⟩}
\newacronym[longplural={尺度不变特征变换(Scale-Invariant Feature Transform, SIFT)}]%
{sift}{SIFT}{尺度不变特征变换(Scale-Invariant Feature Transform, SIFT)}
\newcommand{\sift}{\gls*{sift}}

\newacronym{scale}{尺度不变性}{尺度不变性(scale invariance)}
\newcommand{\sinvariance}{\emph{\gls*{scale}}}

\newacronym{rotation}{旋转不变性}{旋转不变性(rotation invariance)}
\newcommand{\rinvariance}{\emph{\gls*{rotation}}}

\newacronym{gspd}{高斯偏导}{高斯偏导(Derivative of Gaussian)}
\newcommand{\gspd}{\gls*{gspd}}

\newacronym{lpl}{LoG}{拉普拉斯核(Laplacian of Gaussian, LoG)}
\newcommand{\lpl}{\gls*{lpl}}

\newacronym{DoG}{DoG}{高斯差分(Difference of Gaussian, DoG)}
\newcommand{\DoG}{\gls*{DoG}}

\newacronym{octave}{octave}{octave}
\newcommand{\octave}{\emph{\gls*{octave}}}
